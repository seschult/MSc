\documentclass[10pt]{article} 
\textwidth15.0cm

\usepackage{geometry} % Required for adjusting page dimensions

%\longindentation=0pt % Un-commenting this line will push the closing "Sincerely," to the left of the page

\geometry{
	paper=a4paper, % Change to letterpaper for US letter
	top=4cm, % Top margin
	bottom=1.5cm, % Bottom margin
	left=3.0cm, % Left margin
	right=3.5cm, % Right margin
	%showframe, % Uncomment to show how the type block is set on the page
}

\usepackage{geometry} % Required for adjusting page dimensions

%\longindentation=0pt % Un-commenting this line will push the closing "Sincerely," to the left of the page



\begin{document}
	

\noindent Dear Sir or Madam,
\vspace{1.0cm}

\noindent I am currently completing my masters degree at the Max-Planck-Institute for Physics (MPP) in Munich, which I intend to finish in January 2018. In the following, I would like to start a PhD in the field of high energy physics and with this letter I would like to apply for the open position in your institute.\\

\noindent I received my bachelor of physics in 2015 from the Technical University of Munich, after I had finished an apprenticeship as electronics technician at Deckel MAHO, which is one of the worlds leading companies in terms of mechanical engineering. During my masters courses, I have gotten excited by the field of elementary particle physics very quickly and chose courses and seminars, e.g. Testing the Standard Model of Particle Physics, Computational Physics, Gas Detectors: Theory and Application, Particle Physics with Cosmic and Terrestic Particle Colliders and Satellite-Based Particle Physics, to get a well-rounded education in Quantum Field Theory, Detector Physics and Data Analysis.\\

\noindent In parallel to my studies at the university, I also worked as a trainee in several fields of physics, e.g. in neutron scattering at the research reactor FRM2 in Garching or at the ATLAS-MDT group of the MPP. At the MPP I contributed to the hardware development, where I could apply my technical experience. In addition I gained experience in the application of data analysis tools, which I have been using for my masters project, the top-quark mass measurement in the lepton + jets channel at 13 TeV with ATLAS. This work is based on the previous analysis for 7 and 8~TeV \cite{Aad:2015nba,ATLAS:2017lqh} \\

\noindent Furthermore, I also took part in this years (2017) summer student program at CERN, where I could improve my knowledge about theoretical particle physics and computational physics by attending lectures about effective field theories, phenomenological particle physics and statistical methods. My summer student project was the extraction of the strong coupling constant $\alpha_s$ from photon-structure functions $F_2^{\gamma}$ (see~\cite{Albino:2002ck}). The main goal of this summer student project has been the implementation of the $F_2^{\gamma}$ fit and the extraction of $\alpha_s$ in the existing proton PDF fit framework xFitter~\cite{Alekhin:2014irh}. In terms of this project I have learned a lot about QCD, e.g. about the evolution of the PDFs via the DGLAB equations. I am still involved in this project to date.\\

\noindent For my prospective work, my interests are based on the contribution for future analysis, as well as in the development of new experimental equipment to approach the challenges which will arise at increasing luminosity.\\
\noindent Particularly high precision measurements and the investigation of CP-violation seem very interesting to me, since they could be the cornerstone for the discovery of new physics beyond the Standard Model. Therefore, the latest results of the LHCb collaboration are very fascinating. A very good example is the deviation from the Standard Model predictions, observed in the angular distribution of the  $B^0\rightarrow K^{\ast 0}\mu\mu$ channel~\cite{Aaij:2015oid} .\\
\noindent In addition to this, I am also really enthusiastic about the  the first observations of top-quarks in the forward direction~\cite{Aaij:2015mwa}. These fantastic and promising  measurements, have deeply influenced my scientific interests,  thus I would like to contribute in the future LHCb analysis tasks.\\
\noindent Furthermore, I am also very interested in the technical details. The LHCb experiment has, due to  its special conditions and needs, a very unique detector system. The chance to contribute in the future development, e.g.  of the hardware and trigger system, are very motivating, especially if I consider the challenges, which  arise with the upcoming high luminosity LHC (HL-LHC). In order to handle the fast amount of data, which will occur with the LHC upgrade, new technical  and exciting solutions have to be found.\\
\clearpage
\noindent I would be horned to contribute in that work, especially at a renowned university like yours, which offers in my opinion great possibilities, due the fact that you are strongly involved in the development of the LHCb detector system and in the data analysis. \\ 
\noindent Since the EPFL Lausanne is a partner university of Technical University of Munich and also a member of the EuroTech University alliance, I have gotten a good impression about the excellent capabilities you can offer. Therefore, I would be grateful to do my PhD at your university.\\

\noindent If you have further questions, please do not hesitate to contact me.\\
\vspace{0.5cm}

\noindent Yours sincerely,\\

\vspace{0.5cm}
\noindent Sebastian Schulte}


\vspace{0.5cm}
 

\begin{thebibliography}{99}
	

	\bibitem{Aad:2015nba} 
	G.~Aad {\it et al.} [ATLAS Collaboration],
	%``Measurement of the top quark mass in the $t\bar{t}\rightarrow \text{ lepton+jets } $ and $t\bar{t}\rightarrow \text{ dilepton } $ channels using $\sqrt{s}=7$   ${\mathrm { TeV}}$ ATLAS data,''
	Eur.\ Phys.\ J.\ C {\bf 75}, no. 7, 330 (2015)
	doi:10.1140/epjc/s10052-015-3544-0
	[arXiv:1503.05427 [hep-ex]].
	%%CITATION = doi:10.1140/epjc/s10052-015-3544-0;%%
	%81 citations counted in INSPIRE as of 27 Nov 2017
		%\cite{ATLAS:2017lqh}
	\bibitem{ATLAS:2017lqh} 
	The ATLAS collaboration [ATLAS Collaboration],
	%``Measurement of the top quark mass in the $t\bar{t}$→ lepton+jets channel from $\sqrt{s}$=8 TeV ATLAS data,''
	ATLAS-CONF-2017-071.
	%%CITATION = ATLAS-CONF-2017-071;%%
	%1 citations counted in INSPIRE as of 27 Nov 2017
	%\cite{Aad:2015nba}
	%\cite{Alekhin:2014irh}
		\bibitem{Albino:2002ck} 
	S.~Albino, M.~Klasen and S.~Soldner-Rembold,
	%``Strong coupling constant from the photon structure function,''
	Phys.\ Rev.\ Lett.\  {\bf 89}, 122004 (2002)
	doi:10.1103/PhysRevLett.89.122004
	[hep-ph/0205069].
	%%CITATION = doi:10.1103/PhysRevLett.89.122004;%%
	%33 citations counted in INSPIRE as of 27 Nov 2017
	%\cite{Aaij:2015mwa}
	\bibitem{Alekhin:2014irh} 
	S.~Alekhin {\it et al.},
	%``HERAFitter,''
	Eur.\ Phys.\ J.\ C {\bf 75}, no. 7, 304 (2015)
	doi:10.1140/epjc/s10052-015-3480-z
	[arXiv:1410.4412 [hep-ph]].
	%%CITATION = doi:10.1140/epjc/s10052-015-3480-z;%%
	%93 citations counted in INSPIRE as of 27 Nov 2017
	%\cite{Albino:2002ck}
	


	\bibitem{Aaij:2015oid} 
	R.~Aaij {\it et al.} [LHCb Collaboration],
	%``Angular analysis of the $B^{0} \to K^{*0} \mu^{+} \mu^{-}$ decay using 3 fb$^{-1}$ of integrated luminosity,''
	JHEP {\bf 1602}, 104 (2016)
	doi:10.1007/JHEP02(2016)104
	[arXiv:1512.04442 [hep-ex]].
	%%CITATION = doi:10.1007/JHEP02(2016)104;%%
	%232 citations counted in INSPIRE as of 27 Nov 2017
	%\cite{Aaij:2014azz}
	\bibitem{Aaij:2014azz} 
	R.~Aaij {\it et al.} [LHCb Collaboration],
	%``Search for the lepton flavour violating decay τ$^{−}$ → μ$^{−}$ μ$^{+}$ μ$^{−}$,''
	JHEP {\bf 1502}, 121 (2015)
	doi:10.1007/JHEP02(2015)121
	[arXiv:1409.8548 [hep-ex]].
	%%CITATION = doi:10.1007/JHEP02(2015)121;%%
	%41 citations counted in INSPIRE as of 27 Nov 2017
	
	
		\bibitem{Aaij:2015mwa} 
	R.~Aaij {\it et al.} [LHCb Collaboration],
	%``First observation of top quark production in the forward region,''
	Phys.\ Rev.\ Lett.\  {\bf 115}, no. 11, 112001 (2015)
	doi:10.1103/PhysRevLett.115.112001
	[arXiv:1506.00903 [hep-ex]].
	%%CITATION = doi:10.1103/PhysRevLett.115.112001;%%
	%32 citations counted in INSPIRE as of 27 Nov 2017
	%\cite{Aaij:2015oid}
	
\end{thebibliography}






  
\end{document}