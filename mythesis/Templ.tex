%------------------------------------------------------------------------------
\chapter{Measurement of Top-Quark Mass}
\label{sec:Temp1}
%------------------------------------------------------------------------------

The event selection and reconstruction are the base for the final analysis step, which is required to determine the mass of the top-quark. In this final step, the template method (see ~\cref{massdef}) is adopted to determine $m_{\text{top}}$  from Monte Carlo (MC) simulated distributions of observables, which are sensitive to the mass value. These MC simulated samples are generated for different input values of $m_{\text{top}}$ and parametrized with analytical functions, which represent probability densities. These probabilities are used to determine the top-quark mass in an unbinned maximum likelihood fit.\\

\noindent The template method is a standard analysis tool in high energy physics. However, there are certain issues affecting the measurement of the top-quark mass.  Hence, a unique approach has been developed and applied very successfully for the 7 and 8~TeV measurements~\cite{Aad:2015nba,ATLAS-CONF-2017-071}.\\

\noindent In the following the complete template parametrization is discussed in detail. Furthermore, the likelihood fit and the necessary analysis related assumptions are explained. Finally, the consistency of the fit procedure is tested. 


\section{3D-Template Method}
The measurement of the top-quark mass with the template method, suffers from large uncertainties, which arise from the jet energy scale (JES) and the $b$-to-light jet energy scale (bJSE), which is related to energy calibration of $b$-jets. The large contributions of the scale factors to the systematic uncertainties,  are shown by the results of the previous $m_{\text{top}}$ measurements. For  the 7~TeV analysis, the JES uncertainty is with 0.58 $ \pm $ 0.11~GeV ~\cite{Aad:2015nba} one of the largest systematics, as well as for  the 8~TeV measurement, where the JES uncertainty is 0.54 $\pm$ 0.11~GeV~\cite{ATLAS-CONF-2017-071}.\\

\noindent The basic idea to face these large jet energy scale uncertainties is to measure the top-quark mass simultaneously with the jet energy scale factor and the $b$-to-light jet energy scale factor. The simultaneous determination of these three variables allows to absorb the effects arising from energy scale variations, by making use of the multidimensionality of the fit. 
However, in order to make use of the higher dimensionality of the fit, one needs enough statics thus to consider the sum in quadrature of the systematic uncertainties as small, compared to JSE and bJSE uncertainties~\cite{ATLAS-CONF-2017-071}. \\  


\clearpage


\noindent The success of the three dimensional approach can be seen by the remarkable results from the 8~TeV measurement. There, the template fit is performed for one, two and three dimensions, which demonstrates the  effect of the higher dimensional fit on the corresponding uncertainties. The achieved improvement of a two dimensional fit, i.e. the determination of $m_{\text{top}}$ and JSF, compared to a one dimensional fit where only $m_{\text{top}}$ is measured, is  8.3~\% of the final error on $m_{\text{top}$~\cite{ATLAS-CONF-2017-071}.
Moreover, the additional measurement of  bJSF improves the total uncertainty about 43~\%~\cite{ATLAS-CONF-2017-071}.
\noindent Despite these impressive results, one has to keep in mind that the situation might be different for the 13~TeV analysis and a qualified statement can only be given after the first evaluation of all uncertainties.\\

\noindent The observables for the 3D template approach have to assure that the sensitivities to all three variables ($m_{top}$, JSF and bJSF) are given. Therefore, the 
 chosen estimators are reconstructed observables from \textsc{KLFitter}, which  have already been introduced in~\cref{ch5}.
The first estimator is the reconstructed top-quark mass $m_{\text{top}}^{reco}$, which is obtained from the kinematic likelihood fit of the event reconstruction.
From the corresponding  jet permutation of the \textsc{KLFitter} output, the reconstructed $W$-boson mass $m_{\text{W}}^{reco}$ is used, as well as the third variable $R_{\text{bq}}^{reco}$. While $m_{\text{top}}^{reco} $ is determined as free parameter of the kinematic likelihood fit,  the other two are built from the original four vectors.  $R_{\text{bq}}^{reco}$ is here calculated from the two $b$-tag definition. For all three of these reconstructed estimators, Monte Carlo simulated distributions are generated, with an input top-quark mass value between 170.0 and 175.0~GeV. The nominal sample has a mass of 172.5~GeV. Furthermore, all of these distributions are calculated for different scale factors. Therefore, the JSF value is varied in a range between 0.96 and 1.04, while the bJSF value is set to 1.00 and vice versa. \\

\noindent In~\cref{fig:Comparison} the sensitivities of the estimator distributions to different values for $m_{\text{top}}$, JSF and bJSF are shown and compared to 
the nominal sample (red). For each sample, only one variable is varied, while the rest are set to the nominal values, which is 172.5~GeV for $ m_{\text{top}}$ and 1.00 for both scale factors. The deviations from the nominal samples are displayed in the ratio plot below. 
As it can be seen in~\cref{fig:mtopmtop,fig:mtopJSF,fig:mtopbJSF}, the estimator $m_{\text{top}}^{reco}$  shows a noticeable sensitivities to all three variables, while $m_{\text{W}^{reco}$ only varies with the jet energy scale factor (JSF) (see~\cref{fig:mwmtop,fig:mwJSF,fig:mwbJSF}).  The third observable $R_{\text{bq}}^{reco}$ (~\cref{fig:Rbqmtop,fig:RbqJSF,fig:RbqbJSF}), shows a weak sensitivity to the JSF variations, compared to its dependency on the $b$-to-light jet energy scale factor (bJSF). The scale factors are  multiplicative quantities and reduce or increase the number of events  passing the event selection cuts. Events with a higher scale factor, compared to the nominal one, are more likely to pass the event selection than those of samples with a lower scale factor.\\

 \noindent The displayed distributions in~\cref{fig:Comparison} are already normalized to one and parametrized with analytical functions, which are basically the sum of  gaussians and landau distributions. From the parametrized templates, the probability density functions for the unbinned maximum likelihood fit are constructed. \\
 
 

\noindent The parametrization of the templates is one of the major tasks of this thesis. With the new analysis framework, also this part has been implemented completely new, based on the method of the previous measurements. In the following the different steps of the template parametrizations are introduced together with corresponding results. If not mentioned otherwise, all shown distributions are normalized to one. \\



\begin{landscape}
	

\begin{figure} % "[t!]" placement specifier just for this example
	\centering 
	

	\begin{subfigure}{0.37\textwidth}
		\includegraphics[width=\linewidth]{Pics/PlotCombi/mtop_mtop.png}
		\caption{$m_{top}^{reco}$ vs $m_{top}$} \label{fig:mtopmtop}
	\end{subfigure}
	\hspace*{0.25cm}
	\begin{subfigure}{0.37\textwidth}
	\includegraphics[width=\linewidth]{Pics/PlotCombi/mtop_JSF.png}
	\caption{$m_{top}^{reco}$ vs JSF} \label{fig:mtopJSF}
	\end{subfigure}
	\hspace*{0.25cm}
	\begin{subfigure}{0.37\textwidth}
	\includegraphics[width=\linewidth]{Pics/PlotCombi/mtop_bJSF.png}
	\caption{$m_{top}^{reco}$ vs bJSF} \label{fig:mtopbJSF}
	\end{subfigure}
	\begin{subfigure}{0.37\textwidth}
	\includegraphics[width=\linewidth]{Pics/PlotCombi/mw_mtop.png}
	\caption{$m_{W}^{reco}$ vs $m_{top}$} \label{fig:mwmtop}
	\end{subfigure}
	\hspace*{0.25cm}
	\begin{subfigure}{0.37\textwidth}
	\includegraphics[width=\linewidth]{Pics/PlotCombi/mw_JSF.png}
	\caption{$m_{W}^{reco}$ vs JSF} \label{fig:mwJSF}
	\end{subfigure}
	\hspace*{0.25cm}
	\begin{subfigure}{0.37\textwidth}
	\includegraphics[width=\linewidth]{Pics/PlotCombi/mw_bJSF.png}
	\caption{$m_{W}^{reco}$ vs bJSF} \label{fig:mwbJSF}
	\end{subfigure}
	\begin{subfigure}{0.37\textwidth}
	\includegraphics[width=\linewidth]{Pics/PlotCombi/rbq_mtop.png}
	\caption{$R_{bq}^{reco}$ vs $m_{top}$} \label{fig:Rbqmtop}
\end{subfigure}
\hspace*{0.25cm}
\begin{subfigure}{0.37\textwidth}
	\includegraphics[width=\linewidth]{Pics/PlotCombi/rbq_JSF.png}
	\caption{$R_{bq}^{reco}$ vs JSF} \label{fig:RbqJSF}
\end{subfigure}
\hspace*{0.25cm}
\begin{subfigure}{0.37\textwidth}
	\includegraphics[width=\linewidth]{Pics/PlotCombi/rbq_bJSF.png}
	\caption{$R_{bq}^{reco}$ vs bJSF} \label{fig:RbqbJSF}
\end{subfigure}
	\caption{Comparison of signal $t\bar{t}$ templates for different simulated top-quark masses and scale factors. For each of the three observables, ($m_{top}^{reco}$, $m_{W}^{reco}$ and $R_{bq}^{reco}$) the sensitivities are displayed, by overlaying the nominal (red) sample with samples for different JSF, bJSF and $m_{top}$ . The unvaried quantity is kept to the nominal value. }
\end{figure}\label{fig:Comparison}	
\end{landscape}
  
  
\clearpage  

\subsubsection{The Template Parametrization}


The parametrization of the estimator distributions, consists of several sub-steps. Firstly, the actual parametrization, which is used to obtain the probability densities, is performed. The procedure is briefly as follows: 
\begin{itemize}
	\item First of all, each single distribution is fitted with the analytical functions described below to obtain a set of parameters $p_i$ (with $i$ =0,1,...).
	\item The fitted parameters $p_i$ are parametrized linearly as functions of $m_{\text{top}}$, JSF and bJSF. 
	\item With parameters from the two steps above, as initial values, a combined fit is performed to obtain the final parametrization of the templates.
\end{itemize}   

 
\noindent In case of this theses,  only templates of signal events are used to perform a first test of the method. However, the major changes, which have to be considered for the inclusion of the background samples are also taken into account in following.\\  


\subsubsection{The Single Fits}  


\noindent At the beginning of the parametrization process, the best possible initial conditions are searched for the  combined fit of the templates. Furthermore, one is also interested in the dependencies between the three estimators ($m_{\text{top}}^{reco}$, $m_{\text{W}}^{reco}$ and $R_{\text{bq}}^{reco}$) and the three observables, which should be determined ($m_{\text{top}}$,JSF and bJSF). Therefore, two interpolation steps are performed. (For this, the analysis framework of the 8~TeV measurement is used.)\\

\noindent First of all, each single reconstructed template is fitted with a  analytical function. For  $m_{\text{top}}^{reco}$  and  $m_{\text{W}}^{reco}$, this is the sum of a gaussian, a normal landau function and a mirrored landau function.  $R_{\text{bq}}^{reco}$ is fitted with the sum of two gaussians and one landau distribution. 
Since $m_{W}^{reco}$ does not show any significant sensitivities on $m_{\text{top}}$ and bJSF, only the samples with different JSF are parametrized. 
In terms of this thesis, the functional from of $m_{\text{W}}^{reco}$ parametrization was adopted from a sum of two gaussians to the current one.  This choice was made by the observation of a better description of the $m_{\text{W}}^{reco}$ distribution, with the new introduce functions.\\

\noindent From the fist parametrization step nine parameters $p_i$ are obtained for each fitted sample of all three distributions ($m_{\text{top}}^{reco}$, $m_{\text{W}}^{reco}$ and $R_{\text{bq}}^{reco}$). The  parameters $p_i$ represent the constants of the analytical functions, i.e. the amplitude, the mean and the standard deviation. The detailed assignment of the functions can be find in~\cref{sec:apptem} in~\cref{tab:parameters}. Each of these parameter sets, with the nine parameters are further parametrised. Therefore, all parameters of the different sets with same index $i$  are plotted against the simulated $m_{top}$, JSF and bJSF of the corresponding sample. The plots are parametrized with linear functions.  Each parameter $p_i$ can than be written in the following from:
\begin{eqnarray}
&&p_i = a_{0i} + a_{1i} m_{\text{top}}, \nonumber \\  
&& p_i = b_{0i} + b_{1i}\text{JSF}      \hspace{0.5cm}     \text{and}      \nonumber \\   
&& p_i = c_{0i} + c_{1i}\text{bJSF}.     
\end{eqnarray}
The parameters $p_i$ are described by the slope and the offset of the linear functions. All parameters which are determined in these two steps  are used  as initial parameters  for the combined template parametrization.\\

\noindent In~\cref{fig:linear1} examples of the linear parametrization of the fitted parameters, which belong to the single template fit, are shown. The parameters of the different individual $m_{\text{top}}^{reco}$ template fits are fitted against the simulated top-quark mass, and the different JSF and bJSF.  In analogy, there are linear parametrizations of the same kind for  $R_{\text{bq}}^{reco}$, while for $m_{\text{W}}^{reco}$ there is only a linear fit in terms of the jet energy scale JES. The linear dependencies between the parameters of the estimator fits and the observables are essential for the analysis. This will become more clear below with the introduction of the simultaneous parametrization.Only if the distributions are well described and the linearity of each parameter has been achieved, the results of the single fits can be used to perform the simultaneous fit to the estimator distribution.\\ 

\noindent The complete linear fit of all  parameters and the corresponding plots can be find in~\cref{sec:apptem}. In order to achieve a satisfying result, some parameters are fixed, this can be seen, by the fact that the linear fit results in a constant straight line.\\




\begin{figure} % "[t!]" placement specifier just for this example
	\centering 
	
	
	\begin{subfigure}{0.45\textwidth}
		\includegraphics[width=\linewidth]{Pics/Linear/{Lin0_d0_p2_n}.png}
		\caption{$m_{\text{top}}^{\rm reco}$ P2 vs $m_{\text{top}}$} \label{fig:lin1}
	\end{subfigure}
	\hspace*{0.2cm}
	\begin{subfigure}{0.45\textwidth}
		\includegraphics[width=\linewidth]{Pics/Linear/{Lin0_d1_p1_n}.png}
		\caption{$m_{\text{W}}^{\rm reco}$ P1 vs JSF} \label{fig:lin2}
	\end{subfigure}

	\begin{subfigure}{0.45\textwidth}
	\includegraphics[width=\linewidth]{Pics/Linear/{Lin0_d2_p3_n}.png}
	\caption{$R_{\text{bq}}^{\rm reco}$ P3 vs bJSF} \label{fig:lin3}
\end{subfigure}
	\caption{Linear parametrization $p_i$ of the parameters, obtained by the single template fits. Selected parameters are fitted against the different simulated values of $m_{\text{top}}$ and the scale factors. For the scale factors, only the differences $\Delta$ are displayed.}\label{fig:linear1}
\end{figure}



\subsubsection{Simultaneous Template Parametrization }

With the parameters obtained by the previous parametrization steps, a final simultaneous fit to all the different reconstructed distributions is performed. Therefore, the same approach as for the previous analyses is implemented in completely new  fit framework, which has been one of the main tasks of this master project.\\


\noindent The key ingredients for the combined parametrization of the templates are the linear dependencies, between the different parameters of the fit functions and the three variables $m_{\text{top}}$, JSF and $b$JSF, which are used  for a sup-parametrization of the parameter set  $p_i$ of the analytical fit functions:
\begin{equation}\label{eq:lin}
p_i = a_{0i} + a_{1i}(m_{\text{top}}-172.5) + a_{2i}(\text{JSF}-1.00) + a_{3i}(\text{bJSF}-1.00) .
\end{equation}
For the reconstructed $W$-boson mass, this  is adopted by neglecting the parametrization in  $m_{\text{top}}$ and bJSF. The used functional forms are exactly the same as used for the parametrization of the single templates.  With the introduced sub-parametrization  the $\chi^2$ for each single template (for all masses, all JSFs and all bJSFs), is calculated:
\begin{equation}
\chi^2_k = \sum_{n} \left( \frac{h_{k}(x_n)-f_{k}(x_n)}{\sigma_{x_n}}\right)^2.
\end{equation}
The different samples are denoted by $k$ (e.g. $m_{\text{top}}^{reco}$ with $m_{\text{top}}$ = 172.5~GeV, JSF = 1.00 and bJSF = 1.00). The sum runs over all entries $n$ of the corresponding histogram $h_k$, e.g. $m_{\text{top}}^{reco}$. $x_n$ denotes the number of events of $h_k$ at entry $n$. $\sigma(x_n)$ is the corresponding error obtained for $x_n$.  After the $\chi^2$ for each single distribution is calculated, a total $\chi^{2~j}_{\text{tot}}$ for set of each estimator distribution  $j$ = $m_{\text{top}}^{reco}$, $m_{\text{W}}^{reco}$ and $R_{\text{bq}}^{reco}$ is summed up:
\begin{equation}
\chi^2_{\text{tot}}^{j} = \sum_{i}  \chi^2_{i}^{j}.
\end{equation} 
These  are finally minimized with the \textsc{MINUIT} framework~\cite{James:2004xla} and one ends up with 36 parameters in total, for each of the three distributions.\\




\noindent The template fit for each distribution is plotted, together with the corresponding fit functions as shown e.g. in~\cref{fig:mtoptemp1,fig:mtoptemp2,fig:mtoptemp3}. It is important to keep in mind, that the displayed parametrizations are not obtained via a single fit of each distribution. Instead the results show the parametrization of the individual distributions under the simultaneous fit in $m_{\text{top}}$, JSF and bJSF. In this case, the parameters of the different templates are correlated. Further parametrized templates are shown in~\cref{sec:apptem}. The individual templates show, that the chosen fit functions describe the distributions well, which is in good agreement with the observations from the previous measurements.\\   


\noindent For the simultaneous parametrization of $m_{\text{top}}^{reco}$, $m_{\text{W}}^{reco}$  and $R_{\text{bq}^{reco}$ the knowledge of the dependencies on $m_{\text{top}}$, JSF and bJSF are essential, since the simultaneous parametrization is performed with~\cref{eq:lin}, under the assumption that all  are linear. 
These assumptions have been explicitly proofed by the previous analyses~\cite{Aad:2015nba,ATLAS-CONF-2017-071}  and also by theoretical studies~\cite{Heinrich:2017bqp}. In addition to the linear fits, which are performed her, one can also argue that the linearity can be shown by the simultaneous fit itself.  If the linearity assumption is not true, one would expect to see disagreements in the description  templates by the analytical functions, e.g. for different scale factors. Furthermore, the consistency of the complete analysis and therefore also for the parametrization is here checked via closure tests (see~\cref{ct}).\\ 

\clearpage


\begin{figure} % "[t!]" placement specifier just for this example
	\centering
	
	
	\begin{subfigure}{0.45\textwidth}
		\includegraphics[width=\linewidth]{Pics/Para/{172.501.001.00mtop}.png}
		\caption{Nominal distribution for $m_{\text{top}}^{reco}$.} \label{fig:mtoptemp1}
	\end{subfigure}
\hspace*{0.3cm}
	\begin{subfigure}{0.45\textwidth}
		\includegraphics[width=\linewidth]{Pics/Para/{172.501.001.00mw}.png}
		\caption{Nominal distribution for $m_{\text{W}}^{reco}$.} \label{fig:mtoptemp2}
	\end{subfigure}

	
	\begin{subfigure}{0.45\textwidth}
		\includegraphics[width=\linewidth]{Pics/Para/{172.501.001.00rbq}.png}
		\caption{Nominal distribution for $R_{\text{bq}}^{reco}$.} \label{fig:mtoptemp3}
	\end{subfigure}
	
	\caption{Signal templates for the observables of the nominal sample. In the combined fit of all templates, the $\chi^2$ for each distribution is calculated. The presented results show the agreement of the functional forms for the nominal templates, obtained by the simultaneous minimization.}
\end{figure}	



 
 




\section{The Unbinned Maximum Likelihood Fit}
From the parametrized templates, probability density functions are obtained for the signal $m_{\text{top}}^{reco}$, $m_{\text{W}}^{reco}$  and $R_{\text{bq}}^{reco}$ distributions. These densities are used in an unbinned maximum likelihood fit for all events $i = 1,..,N$, to obtain simultaneously the top-quark mass and the two jet energy scale factors. \\

\noindent The main assumption for the construction of the likelihood function is that the correlations between the three observables are small. In this case the probability  functions can be factorized. This allows to use the product of single probability distributions, while otherwise a multidimensional probability function would have to be taken into account.\\



 \noindent For the illustration of the factorization theorem, one can consider a normal two dimensional probability density function of two variables $x_1$ and $x_2$:
 \begin{equation}
 	P(x_1,x_2) = \frac{1}{2\pi \sigma_1 \sigma_2 \sqrt{1-\rho^2}}  e^{-\frac{1}{2(1-\rho^2)} [\frac{x_1^2}{\sigma_1^2} + \frac{x_2^2}{\sigma_2^2} - \frac{2\rho x_1 x_2}{\sigma_1 \sigma_2}]}.
 \end{equation}
 \noindent This is a two dimensional gaussian, where the correlation between the variables are taken into account by the correlation factor $\rho$. In case of small correlations between two variables, which results in small values for $\rho$, the last term of the distribution can be neglected and the multiplication factor of the exponent becomes -1/2. Thus the probability density can be split into to the product of two single guassian distributions, which reduces the complexity of the functional from.\\
 \begin{figure} [h]% "[t!]" placement specifier just for this example
	\centering
	

	\begin{subfigure}{0.35\textwidth}
		\includegraphics[width=\linewidth]{Pics/PlotCombi/mtopmw.png}
		\caption{Correlation: $m_{top}^{reco}$ and $R_{bq}^{reco}$.} \label{fig:1a}
	\end{subfigure}
	\hspace*{0.1cm}
	\begin{subfigure}{0.35\textwidth}
	\includegraphics[width=\linewidth]{Pics/PlotCombi/mtopRbq.png}
	\caption{Correlation:  $m_{top}^{reco}$ and $m_{W}^{reco}$.} \label{fig:1b}
	\end{subfigure}

	\begin{subfigure}{0.35\textwidth}
	\includegraphics[width=\linewidth]{Pics/PlotCombi/mwRbq2.png}
	\caption{Correlation:  $m_{W}^{reco}$ and $R_{bq}^{reco}$.} \label{fig:1c}
	\end{subfigure}

	\caption{Pair-wise correlation for the observables.}
\end{figure}	

 

\noindent In this measurement, the correlations of all three variables are displayed in~\cref{fig:1a,fig:1b}. The evaluated correlation factors are:
\begin{eqnarray*}
\rho (m_{\text{top}}^{reco} - m_\text{W}^{reco}) = -0.02,\hspace{0.3cm}\rho (m_{\text{top}}^{reco} - R_{\text{bq}}^{reco}) =0.05 \hspace{0.3cm} \text{and} \hspace{0.3cm}
\rho (m_{\text{W}}^{reco} - R_{\text{bq}}^{reco}) = -0.10.\nonumber \\ 
\end{eqnarray*}
\noindent The observed values are sufficiently small, which allows the application of the factorization theorem. \\
 The final likelihood function is:
\begin{eqnarray}
\Likeljets(\mt, \JSF, \bJSF)  &=& 
\prod_{i=1}^{N} P_{\text{top}}(m_{\text{top}}^{reco,i}\,\vert\,\mt, \JSF, \bJSF)\cdot \nonumber \\ &&\quad\quad P_W(m_{\text{W}}^{reco,i}\,\vert\,\JSF)\cdot \nonumber \\ 
 \vspace{0.1cm}
&&\quad\quad P_{R_{\text{bq}}}(R_{\text{bq}}^{reco,i}\,\vert\,\mt,\JSF,\bJSF).
\label{eq:LikeLJ1} 
\end{eqnarray}

\noindent The likelihood is a product of probability density functions $P_k$  with $k = $ top, W, R$_{\text{bq}}$, which runs over all $N$ events. The shapes of the three probability densities are obtained from the functional forms of the template parametrization. The simultaneous template fit parametrizes the functions in $m_{\text{top}}$,  JSF and bJSF. For the constants of these probability density functions (amplitude, mean and standard deviation) the parameter sets, which result from the simultaneous template fit, are used.  $P_{\text{top}}(m_{\text{top}}}^{reco,i}\,\vert\,\mt, \JSF, \bJSF)$ corresponds to the  $m_{\text{top}}^{reco}$ distribution, which is parametrized in $\mt,\JSF $ and $\bJSF$ like $P_{R_{\text{bq}}}(R_{\text{bq}}^{reco,i}\,\vert\,\mt,\JSF,\bJSF)$, the probability density function of $R_{\text{bq}}^{reco}$. 
The probability density function of the $m_{\text{W}}}^{reco}$ distribution  $P_{\text{W}}(m_{\text{W}}}^{reco,i}\,\vert\, \JSF)$ is only parametrized in JSF.  In  this thesis, the probability density functions are representing the signal distributions, without the single top-quark samples. With the minimization of the likelihood, $m_{\text{top}$, JSF and bJSF are determined, which is in the following called three dimensional fit, respectively two and one dimensional fits denote the determination of 
$m_{\text{top}$ and JSF or $m_{\text{top}$ only.\\


\noindent For the previous top-quark mass measurements in the lepton + jets channel, also the background templates are parametrized and taken into account by additional probability functions. In addition to $m_{\text{top}}$, JSF and bJSF,  also the background fraction $f_{bkg}$ is determined with the maximum likelihood fit.  In that case, the corresponding probability density functions take the following from: 

\begin{eqnarray}
	P_{top}(\mtri\,\vert\,\mt, \JSF, \bJSF, \fbkg) &=& 
	(1-\fbkg)\cdot\Ptopsig(\mtri\,\vert\,\mt, \JSF, \bJSF) +
	\nonumber \\&&\quad\quad\,
	\fbkg\cdot\Ptopbkg(\mtri\,\vert\,\JSF, \bJSF)\,\,,
	\nonumber \\
	P_W(m_{\text{W}}^{reco,i}\,\vert\,\JSF, \fbkg) &=& (1-\fbkg)\cdot P_{m_{\text{W}}}^{\text{sig}}(m_{\text{W}}^{reco,i}\,\vert\,\JSF) +
	\nonumber \\
	&& \quad\quad\, \fbkg\cdot P_{\text{W}}^{\text{bkg}}(m_{\text{W}}^{reco,i}\vert\,\JSF)
	\,\,,\nonumber \\
	P_{R_{\text{bq}}}(\rlbri\,\vert\,\mt,JSF,\bJSF, \fbkg)
	&=& (1-\fbkg)\cdot P_{R_{\text{bq}}}^{\text{sig}}(\rlbri \,\vert \mt,\JSF,\bJSF) +
	\nonumber \\
	&& \quad\quad\, \fbkg\cdot P_{R_{\text{bq}}}^{\text{bkg}}(\rlbri\,\vert\,\bJSF).\nonumber \\
		\nonumber \\
\end{eqnarray}




\section{Closure Tests}\label{ct}
The Method performance and the consistency of the estimated properties are tested via closure tests. Ensembles for each parameter set are drawn with pseudodata, generated by randomly chosen events from the available Monte Carlo simulation. Each of the generated ensembles corresponds to the integrated luminosity of the data. Furthermore, the statistical uncertainties are obtained from the fit. Bootstrapping~\cite{efron1992bootstrap} is applied to resample the events after their selection. Therefore, one event can appear in several ensembles, which makes it necessary to take care about unwanted correlations among the different ensembles by assuring that there is enough Monte Carlo statistics or, like for the previous analyses, apply an oversampling correction~\cite{barlow2000application}. 
However, since in this thesis the whole closure test procedure is performed for the first time with the new analyses framework, the oversampling correction is not applied yet.  \\

 \noindent With the pseudo-data the  analysis is performed to determine the top-quark mass and the jet energy scale factors. The whole procedure, from drawing the pseudodata ensembles to the likelihood fit is performed 250 times. 
 For each ensemble the pseudoexperiments return distributons of the fitted estimators. The corresponding expactation values of these distributions are calculated and used to test the linearity of the fit. Therefore, the pseudoexperiments are performed in one, two and three dimensions for five different input values: $m_{\text{top}}^{in}$, JSF$^{in}$ and bJSF$^{in}$. While one input is varied, the other two variables are set to their nominal values. 
For the three dimensional fit all three variables are varied, in two dimensions, only $m_{\text{top}}^{in}$ and JSF$^{in}$ are considered and in the one dimensional fit only  $m_{\text{top}}^{in}$ is varied. \\
  
 \noindent  Closure tests are performed in each dimension. The difference between the input and the output value of the three variables, i.e.  $m_{\text{top}}^{fit}-m_{\text{top}}^{in}$  (analogous for JSF and bJSF) , are  plotted vs the different input values for $m_{\text{top}}^{in}$, JSF$^{in}$ and bJSF$^{in}$. Moreover,  a linear fit is performed, which takes the statistical uncertainties per parameter into account. The slope of the linear fit is forced to be zero. If the offset of the linear fit is consistent with zero, the result is unbiased.\\
 
 %The consistency of the method is further check by performing cross-checks of the closure tests, i.e. the mass residual is also plotted vs JSF$^{in}$, in the 2D fit and vs bJSF$^{in}$ for the 3D case.\\
 
 
 \noindent In addition to the linearity tests, the so-called pull distributions are evaluated by taking also into account the uncertainty for each pseudo-experiment, e.g. $\delta_{m_{\text{top}}^{fit}$. The pull values for $m_{\text{top}}$ are calculated with:  
 \begin{equation}
 m_{\text{top pull}} = \frac{m_{\text{top}}^{in} - m_{\text{top}}^{fit}}{\delta_{m_{\text{top}}}^{fit}}.
 \end{equation}
 The equation can easily be adopted for the JSF and the bJSF by replacing the corresponding quantities. The pull distribution is evaluated for each variation of the three input variables. If the uncertainties $\delta^{fit}$ are correctly estimated, the width of the pull distributions should be close to unity. Therefore, the  pull-width of each distribution is plotted as a function of the input values ($m_{\text{top}}^{in}$, JSF$^{in}$ and bJSF$^{in}$). A straight line fit provides  the quality of the error estimations. If the uncertainties are underestimated, i.e. the average pull width is to small, the fitted line has a positive offset. Hence, if a negative offset is observed  the  average pull-width is larger than one and  the uncertainties are overestimated.\\ 

\noindent The results of the one dimensional closure test can be seen, together with the corresponding pull-width plot (right), in~\cref{closure3}. For the linearity test on the left, only $m_{\text{top}}$ is fitted, while both scale factors are set to 1.00.  The result of the strait line fit on the left, demonstrates a good consistency of the fit in one dimension. The corresponding pull-width fit, however, shows a noticeable offset from unity towards larger values. Thus, the average pull-width for the different mass points is larger than one and the uncertainties are underestimated.




.



\begin{figure}[h] % "[t!]" placement specifier just for this example
	\centering 
		\begin{subfigure}{0.37\textwidth}
	\includegraphics[width=\linewidth]{Pics/{Closure_mtop_diff_250PE_1D}.pdf}
	\label{fig:x1cb}
\end{subfigure}	
\begin{subfigure}{0.37\textwidth}
	\includegraphics[width=\linewidth]{Pics/{Closure_mtop_pull_250PE_1D}.pdf}
	\label{figx:1cb}
\end{subfigure}
\caption{ 
Results of the 1D closure tests. On the right,  $m_{\text{top}}^{fit}$ is plotted vs five different $m_{\text{top}}^{in}$, while  JSF and bJSF are set to 1.00. The results are obtained with 250 pseudoexperiments. In addition, the corresponding pull-widths are shown. The dashed lines represent the ideal value. All error bars correspond to statistical uncertainties. 
}\label{closure3}
\end{figure}


\noindent In~\cref{closure2} the results for two dimensions are presented. In the top line, the closure testes are displayed for the fit of $m_{\text{top}}$ and JSF. bJSF is kept at 1.00. The second line displays the pull-widths. For the two dimensional fit, a similar result as in one dimension is obtained. While the linearity tests of $m_{\text{top}}$ and JSF demonstrate successfully the consistency of the method in two dimensions,  the pull-width shows that the uncertainties are underestimated.\\ 

\noindent The results of the closure tests and the pull-widths for the three dimensional fit of $m_{\text{top}}$, JSF and bJSF are presented in~\cref{closure1}. In contrast to the one and two dimensional closure test, the consistency can not be proven for three dimensions, since the offset in the corresponding plots for $m_{\text{top}}$ and bJSF show that there is  a bias. Furthermore, the offsets, observed for pull-widths show, like for the one and two dimensional case, that the corresponding uncertainties are underestimated.\\

   
\noindent  The reason for the observed bias in three dimensions might arise due to an insufficient description of the reconstructed distributions by the chosen  parametrization of the templates. In the first runs of the closure test, the linearity tests also failed for the one and two dimensional fit. This issue could be solved by optimizing the  initial parameters of the simultaneous template parametrization. Therefore, the  further optimization of the simultaneous fit might improve the result of the 3D closure test as well.  In case of the general underestimation of the uncertainties which is observed for all three dimensions, the corresponding uncertainties have to be studied explicitly. However, it is shown, that the 3D template method is basically consistent at least in one and two dimensions. Therefore, the next analysis step, the evaluation of the systematic uncertainties, takes firstly the reduced number of dimensions into account.


\begin{figure} [h]% "[t!]" placement specifier just for this example
		\centering 

	\begin{subfigure}{0.35\textwidth}
			\includegraphics[width=\linewidth]{Pics/{Closure_mtop_diff_250PE_2D}.pdf}
			\label{fig:12cb}
		\end{subfigure}	
		\hspace*{0.25cm}
		\begin{subfigure}{0.35\textwidth}
			\includegraphics[width=\linewidth]{Pics/{Closure_jsf_diff_250PE_2D}.pdf}
			\label{fig:21cb}
		\end{subfigure}	
		
		\begin{subfigure}{0.35\textwidth}
			\includegraphics[width=\linewidth]{Pics/{Closure_mtop_pull_100PE_2D}.pdf}
			\label{fig:1cm}
		\end{subfigure}
		\hspace*{0.25cm}
		\begin{subfigure}{0.35\textwidth}
			\includegraphics[width=\linewidth]{Pics/{Closure_jsf_pull_100PE_2D}.pdf}
			\label{fig:1cj}
		\end{subfigure}

		
	
		\caption{ 
	The same results as in ~\cref{closure3} are displayed for the two dimensional fit. The top line shows the closure fits for the top-quark mass and JSF. The bottom line displays the corresponding pull distributions.
		}\label{closure2}
	\end{figure}	






\begin{landscape}
\begin{figure} % "[t!]" placement specifier just for this example
	\centering 
	\begin{subfigure}{0.37\textwidth}
	\includegraphics[width=\linewidth]{Pics/{Closure_mtop_diff_250PE_3D}.pdf}
			\caption{3D closure test for $m_{\rm top}^{\rm in}$.}
 \label{fig:3cm}
	\end{subfigure}
	\hspace*{0.25cm}
	\begin{subfigure}{0.37\textwidth}
	\includegraphics[width=\linewidth]{Pics/{Closure_jsf_diff_250PE_3D}.pdf}
			\caption{3D closure test for JSF$^{\rm in}$.}
\label{fig:3cj}
	\end{subfigure}
	\hspace*{0.25cm}
	\begin{subfigure}{0.37\textwidth}
	\includegraphics[width=\linewidth]{Pics/{Closure_bjsf_diff_250PE_3D}.pdf}
			\caption{3D closure test for bJSF$^{\rm in}$.}
\label{fig:3cb}
	\end{subfigure}


	\begin{subfigure}{0.37\textwidth}
	\includegraphics[width=\linewidth]{Pics/{Closure_mtop_pull_250PE_3D}.pdf}
			\caption{3D pull-width for $m_{\rm top}^{\rm in}$.}
\label{fig:3pull1}
	\end{subfigure}
	\hspace*{0.25cm}
	\begin{subfigure}{0.37\textwidth}
	\includegraphics[width=\linewidth]{Pics/{Closure_jsf_pull_250PE_3D}.pdf}
			\caption{3D pull-width for JSF$^{\rm in}$.}
	\label{fig:3pull2}
	\end{subfigure}
	\hspace*{0.25cm}
	\begin{subfigure}{0.37\textwidth}
	\includegraphics[width=\linewidth]{Pics/{Closure_bjsf_pull_250PE_3D}.pdf}
			\caption{3D pull-width for bJSF$^{\rm in}$.}
	 \label{fig:3pull3}
	\end{subfigure}
	
	
	

	\caption{
	The same results as in ~\cref{closure3} are displayed for the three dimensional fit. The top row shows the closure fits for the top-quark mass,  JSF and bJSF. The bottom row displays the corresponding pull distributions.
}\label{closure1}
\end{figure}

	
\end{landscape}







