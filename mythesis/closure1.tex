



\begin{landscape}
\begin{figure} % "[t!]" placement specifier just for this example
	\centering 
	\begin{subfigure}{0.37\textwidth}
	\includegraphics[width=\linewidth]{Pics/{Closure_mtop_diff_250PE_3D}.pdf}
 \label{fig:3cm}
	\end{subfigure}
	\hspace*{0.25cm}
	\begin{subfigure}{0.37\textwidth}
	\includegraphics[width=\linewidth]{Pics/{Closure_jsf_diff_250PE_3D}.pdf}
\label{fig:3cj}
	\end{subfigure}
	\hspace*{0.25cm}
	\begin{subfigure}{0.37\textwidth}
	\includegraphics[width=\linewidth]{Pics/{Closure_bjsf_diff_250PE_3D}.pdf}
\label{fig:3cb}
	\end{subfigure}


	\begin{subfigure}{0.37\textwidth}
	\includegraphics[width=\linewidth]{Pics/{Closure_mtop_pull_250PE_3D}.pdf}
\label{fig:3pull1}
	\end{subfigure}
	\hspace*{0.25cm}
	\begin{subfigure}{0.37\textwidth}
	\includegraphics[width=\linewidth]{Pics/{Closure_jsf_pull_250PE_3D}.pdf}
	\label{fig:3pull2}
	\end{subfigure}
	\hspace*{0.25cm}
	\begin{subfigure}{0.37\textwidth}
	\includegraphics[width=\linewidth]{Pics/{Closure_bjsf_pull_250PE_3D}.pdf}
	 \label{fig:3pull3}
	\end{subfigure}
	
	
	

	\caption{
	The same results as in ~\cref{closure3} are displayed for the three dimensional fit. The top line shows the closure fits for the top-quark mass,  JSF and bJSF. The bottom line displays the corresponding pull distributions.
}\label{closure1}
\end{figure}

	
\end{landscape}






